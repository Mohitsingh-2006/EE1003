
\let\negmedspace\undefined
\let\negthickspace\undefined
\documentclass[journal]{IEEEtran}
\usepackage[a5paper, margin=10mm, onecolumn]{geometry}
%\usepackage{lmodern} % Ensure lmodern is loaded for pdflatex
\usepackage{tfrupee} % Include tfrupee package
\setlength{\headheight}{1cm} % Set the height of the header box
\setlength{\headsep}{0mm}     % Set the distance between the header box and the top of the text
\usepackage{gvv-book}
\usepackage{gvv}
\usepackage{cite}
\usepackage{amsmath,amssymb,amsfonts,amsthm}
\usepackage{algorithmic}
\usepackage{graphicx}
\usepackage{textcomp}
\usepackage{xcolor}
\usepackage{txfonts}
\usepackage{listings}
\usepackage{enumitem}
\usepackage{mathtools}
\usepackage{gensymb}
\usepackage{comment}
\usepackage[breaklinks=true]{hyperref}
\usepackage{tkz-euclide} 
\usepackage{listings}
% \usepackage{gvv}                                        
\def\inputGnumericTable{}                                 
\usepackage[latin1]{inputenc}                                
\usepackage{color}                                            
\usepackage{array}                                            
\usepackage{longtable}                                       
\usepackage{calc}                                             
\usepackage{multirow}                                         
\usepackage{hhline}                                           
\usepackage{ifthen}                                           
\usepackage{lscape}
\renewcommand{\thefigure}{\theenumi}
\renewcommand{\thetable}{\theenumi}
\setlength{\intextsep}{10pt} % Space between text and floats
\numberwithin{equation}{enumi}
\numberwithin{figure}{enumi}
\renewcommand{\thetable}{\theenumi}
\begin{document}
\bibliographystyle{IEEEtran}
\title{11.16.3.7}
\author{EE24BTECH11041 - Mohit}
% \maketitle
% \newpage
% \bigskip
{\let\newpage\relax\maketitle}
\textbf{Question:-} A fair coin is tossed four times, and a person wins Rs.1 for each head and loses Rs.1.50 for each tail. From the sample space, calculate how many different amounts of money you can have after four tosses and the probability of having each of these amounts.

\textbf{Solution}

Let:
\begin{align}
H = +1 \quad \text{(gain Rs.1 for Head)}, \quad T = -1.50 \quad \text{(lose Rs.1.50 for Tail)}.
\end{align}

For $ x $, the number of heads in 4 tosses, the total net money can be calculated using the formula:
\begin{align}
\text{Net Money} = x(1) + (4-x)(-1.5)\\
\text{Net Money} = x - 1.5(4-x)\\
\text{Net Money} = 2.5x - 6,
\end{align}

where $ x = 0, 1, 2, 3, 4 $.

\textbf{Possible Outcomes and Net Money}

\begin{itemize}
    \item $ x = 0 $: All tails ($ TTTT $):
    \begin{align}
    \text{Net Money} = 2.5(0) - 6 = -6
    \end{align}
    \item $ x = 1 $: One head, three tails ($ HTTT, THTT, TTHT, TTTH $, etc.):
    \begin{align}
    \text{Net Money} = 2.5(1) - 6 = -3.5
    \end{align}
    \item $ x = 2 $: Two heads, two tails ($ HHTT, HTHT, HTTH, \dots $):
    \begin{align}
    \text{Net Money} = 2.5(2) - 6 = -1
    \end{align}
    \item $ x = 3 $: Three heads, one tail ($ HHHT, HHTH, HTHH, THHH $):
    \begin{align}
    \text{Net Money} = 2.5(3) - 6 = 1.5
    \end{align}
    \item $ x = 4 $: All heads ($ HHHH $):
    \begin{align}
    \text{Net Money} = 2.5(4) - 6 = 4
    \end{align}
\end{itemize}

\textbf{Number of Outcomes for Each Case}

The number of outcomes for each $ x $ is given by the binomial coefficient $ \binom{4}{x} $:

\begin{align}
    &x = 0: \binom{4}{0} = 1, \\
    &x = 1: \binom{4}{1} = 4, \\
    &x = 2: \binom{4}{2} = 6, \\
    &x = 3: \binom{4}{3} = 4, \\
    &x = 4: \binom{4}{4} = 1
\end{align}

\textbf{Probabilities of Each Case}

Since the coin is fair, the probability of each outcome is $ \frac{1}{16} $. The probabilities for each $ x $ are:

\begin{align}
    &x = 0: \text{Probability} = \frac{\binom{4}{0}}{16} = \frac{1}{16}, \\
    &x = 1: \text{Probability} = \frac{\binom{4}{1}}{16} = \frac{4}{16} = \frac{1}{4}, \\
    &x = 2: \text{Probability} = \frac{\binom{4}{2}}{16} = \frac{6}{16} = \frac{3}{8}, \\
    &x = 3: \text{Probability} = \frac{\binom{4}{3}}{16} = \frac{4}{16} = \frac{1}{4}, \\
    &x = 4: \text{Probability} = \frac{\binom{4}{4}}{16} = \frac{1}{16}.
\end{align}


\textbf{CODING LOGIC:-}


\textbf{Bernoulli Random Variable for Coin Toss}\\
Let the outcome of each coin toss be represented by a Bernoulli random variable \( X \), where:

\[
X = 
\begin{cases} 
1 & \text{if the toss is a head (H)} \\
0 & \text{if the toss is a tail (T)}
\end{cases}
\]

Since the coin is fair, the probability of getting heads or tails is:

\begin{align}
P(X = 1) = P(X = 0) = \frac{1}{2}
\end{align}

\textbf{New Random Variable for Money}\\
Let the random variable \( Y \) represent the amount of money the person wins or loses. For each toss:

\[
Y = 
\begin{cases} 
1 & \text{if the toss is a head (H)} \\
-1.5 & \text{if the toss is a tail (T)}
\end{cases}
\]

\textbf{Money After Four Tosses}\\
Let the number of heads in the four tosses be $ k $, where $ k $ can range from 0 to 4. The total money $ M(k) $ after $k $ heads is the sum of the individual amounts for each toss.

\begin{align}
M(k) = \text{(Amount from Heads)} + \text{(Amount from Tails)} = k \times 1 + (4 - k) \times (-1.5)\\
M(k) = k - 1.5(4 - k) = k - 6 + 1.5k = 2.5k - 6
\end{align}

The possible values of $ M(k) $ are:
- For $ k = 0 $: $ M(0) = -6 $
- For $ k = 1 $: $ M(1) = -3.5 $
- For $ k = 2 $: $ M(2) = -1 $
- For $ k = 3 $: $ M(3) = 1.5 $
- For $ k = 4 $: $ M(4) = 4 $

\textbf{Probability Mass Function (PMF)}\\
The PMF of the random variable \( Y \) gives the probabilities of each possible amount of money:

\begin{align}
P(M = -6) = \frac{1}{16},\\ P(M = -3.5) = \frac{4}{16},\\ P(M = -1) = \frac{6}{16}, \\ P(M = 1.5) = \frac{4}{16}, \\ P(M = 4) = \frac{1}{16}
\end{align}

\textbf{Cumulative Distribution Function (CDF)}\\
The CDF gives the probability that the random variable takes a value less than or equal to a particular amount of money. The CDF values are:

\begin{align}
P(M \leq -6) = \frac{1}{16}\\
P(M \leq -3.5) = \frac{5}{16}\\
P(M \leq -1) = \frac{11}{16}\\
P(M \leq 1.5) = \frac{15}{16}\\
P(M \leq 4) = 1
\end{align}


\begin{figure}[h!]
   \centering
   \includegraphics[width=0.7\linewidth]{figs/Figure_1.png}
\end{figure}

\end{document}
